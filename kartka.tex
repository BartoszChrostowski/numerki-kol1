%Bartosz Chrostowski
%dziękuję współtworzącym 
%2018 sesja letnia

\documentclass[twocolumn]{article}
\usepackage[a5paper,margin=0.7cm, top=0.7cm,bottom=1cm]{geometry}
\usepackage[T1]{fontenc}
\usepackage{lmodern}
\usepackage[polish]{babel} 
\usepackage[utf8]{inputenc}
\selectlanguage{polish}
\usepackage{amsfonts} %\mathbb
\usepackage{sectsty} %\sectionfont
\usepackage{amsmath} %\underset matrix 
\usepackage{algorithmic} % pseudokod
\sectionfont{\large}
\subsectionfont{\normalsize}
\usepackage{scrextend} % ? stackexchange: how to change latex less 10pt
\changefontsizes[9pt]{6pt} %? stackexchange: how to change latex less 10pt
\usepackage{lipsum} %? stackexchange: how to change latex less 10pt
\usepackage[pdftex	,pdfauthor={Bartosz Chrostowski},pdftitle={Kartka na egzamin},pdfsubject={Metody numeryczne 1}]{hyperref}

\begin{document}
\begin{flushleft}
\thispagestyle{empty} %XD

\section{Całkowanie numeryczne}
Kwadratura jest rzędu $r$ jeśli jest dokładna dla wszystkich wielomianów stopnie $r-1$ oraz istnieje wielomian stopnia r dla której nie jest dokładna\\
\subsection{Trapezy $\xi \in \left(a,b\right) \quad E(f) = I(f) - S(f)$}
$$E(f) = -\sum_{k=1}^{N}\frac{H^3f^{\prime\prime}(\xi_k)}{12} =- \frac{(b-a)^3}{12N^2}f^{\prime\prime}(\xi)$$
\subsection{Prostokąty $\xi \in \left(a,b\right) \quad E(f) = I(f) - S(f)$}
$$E(f) = \frac{(b-a)^3}{24N^2}f^{\prime\prime}(\xi) = \frac{H^2}{24}(b-a) f^{\prime\prime}(\xi)$$
\subsection{Simpson $\xi \in \left(a,b\right) \quad E(f) = I(f) - S(f)$}
$$E(f)= -\sum_{k=1}^N \frac{H^5f^{(4)}(\xi_k)}{90\cdot2^5} =- \frac{(b-a)^5}{2880N^4}f^{(4)}(\xi) $$% \leq \frac{H^4(b-a)}{180\cdot2^4}\underset{\xi \in [a,b]}{\sup}|f^{(4)}(\xi)| $$
\subsection{Złożona $S_1$ na $[a,b]$ $S_2$ na $[c,d]$}
$$\underset{y\in[c,d]}{max} \left| S_1(f)(y) - \int_a^bf(x,y)dx \right| \leq e_1$$
$$\underset{x\in[a,b]}{max} \left| S_2(f)(x) - \int_c^df(x,y)du \right| \leq e_2$$
$$\left| S(f) - \int_c^d\int_a^b f(x,y) dx \,dy \right| \leq (d-c) e_1 + (b-a) e_2$$
\subsection{TODO jeszcze inne jakieś tam gaussy.}
Rząd aproksymacji równy jest ilości parametrów które ustalamy (dla gaussa przy liczeniu od 0 mamy $2n+2$)
\section{Aproksymacja}
Konstruujemy macierz $G = \left[\langle g_i,g_j \rangle\right]_{i,k = 0}^n$ (w wierszu nie zmienia się to co na pierwsze w iloczynie) wektor $d = \left[\langle g_i, f_i\rangle \right]_{i = 0}^n$ i szukamy rozwiązania $Ax=d$\\
Dla dyskretnej można $M = \left[f_j(x_i)\right]_{i,k = 0}^n$ (w wierszu te same $x_i$) $G = M^TM$ \ \ \ \  $d = M^TF$ gdzie F to wektor dany z wartościami przybliżanej funkcji
\section{Splajny}
$S'(a)=f'(a)$, $S'(b)=f'(b)$ - warunki Hermite'a\\
$S'(a)=S'(b)$, $S''(a)=S'(b)$ - warunki cykliczne\\
$S''(a)=S''(b)=0$ naturalna funkcja sklejana\\
nota-knot:$
\begin{cases}
    S(\frac{x_0+x_1}{2})=\frac{f(x_0)+f(x_2)}{2}, \\
    S(\frac{x_{n-1}+x_n}{2})=\frac{f(x_{n-1})+f(x_n)}{2},
\end{cases}$
Funkcja sklejana n-tego stopnia to jest, że jest klasy $C^{n-1}$ 
\section{Metoda iteracji prostej}
Tw. $D$ zbiór domknięty w $\mathbb{R}$ oraz niech $g:D\rightarrow D$ i jest odwzorowaniem zwężającym wtedy metoda iteracji prostej $g$ jest zbieżna na $D$ do dokładnie jednego rozwiązania.\\
Jeśli g jest ciągła i $D$ jest domknięty to mamy\\
$g(x_1) - g(x_0) = g^{\prime}(\xi)(x_1 - x_0)$\\
$g^{\prime}$ to macierz Jakobiego\\
ponadto:\\
$||g(x_1) - g(x_0)|| = ||g^{\prime}(\xi)(x_1 - x_0)|| \leq ||g^{\prime}(\xi)||(x_1 - x_0)||$
jeśli $||g^{\prime}(\xi)|| < 1$ to mamy odwzorowanie zwężające.\\
Ograniczając z góry każdy wyraz z macierzy $g^{\prime}$ otrzymamy macierz $A$ i zachodzi $||g^{\prime}(\xi)|| < ||A||$ (jeśli ograniczaliśmy $\leq$ analogicznie)
\end{flushleft}
\thispagestyle{empty}
\end{document}
