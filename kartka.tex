%Bartosz Chrostowski
%dziękuję współtworzącym 
%2018 sesja letnia

\documentclass[twocolumn]{article}
\usepackage[a5paper,margin=0.5cm, top=0.7cm,bottom=1cm]{geometry}
\usepackage[T1]{fontenc}
\usepackage{lmodern}
\usepackage[polish]{babel} 
\usepackage[utf8]{inputenc}
\selectlanguage{polish}
\usepackage{amsfonts} %\mathbb
\usepackage{sectsty} %\sectionfont
\usepackage{amsmath} %\underset matrix 
\usepackage{algorithmic} % pseudokod
\sectionfont{\large}
\subsectionfont{\normalsize}
\usepackage{scrextend} % ? stackexchange: how to change latex less 10pt
\changefontsizes[9pt]{7pt} %? stackexchange: how to change latex less 10pt
\usepackage{lipsum} %? stackexchange: how to change latex less 10pt
\usepackage[pdftex	,pdfauthor={Bartosz Chrostowski et al.},pdftitle={Kartka na kolokwium},pdfsubject={Metody numeryczne 2}]{hyperref}

\begin{document}
\begin{flushleft}
\thispagestyle{empty} %XD XD

\section{Całkowanie numeryczne}
Kwadratura jest rzędu $r$ jeśli jest dokładna dla wszystkich wielomianów stopnia $\leq r-1$ oraz istnieje wielomian stopnia $r$ dla której nie jest dokładna\\
\subsection{Trapezy $\xi \in \left(a,b\right) \quad E(f) = I(f) - S(f)$}
$$E(f) = -\sum_{k=1}^{N}\frac{H^3f^{\prime\prime}(\xi_k)}{12} =- \frac{(b-a)^3}{12N^2}f^{\prime\prime}(\xi)$$
\subsection{Prostokąty $\xi \in \left(a,b\right) \quad E(f) = I(f) - S(f)$}
$$E(f) = \frac{(b-a)^3}{24N^2}f^{\prime\prime}(\xi) = \frac{H^2}{24}(b-a) f^{\prime\prime}(\xi)$$
\subsection{Simpson $\xi \in \left(a,b\right) \quad E(f) = I(f) - S(f)$}
$$E(f)= -\sum_{k=1}^N \frac{H^5f^{(4)}(\xi_k)}{90\cdot2^5} =- \frac{(b-a)^5}{2880N^4}f^{(4)}(\xi) $$% \leq \frac{H^4(b-a)}{180\cdot2^4}\underset{\xi \in [a,b]}{\sup}|f^{(4)}(\xi)| $$
\subsection{Złożona $S_1$ na $[a,b]$ $S_2$ na $[c,d]$}
$$\int_c^d\int_a^b f(x,y) dx\, dy \approx S_2(S_1(f)) =$$
$$=\sum_{j=0}^m B_j (\sum_{i=0}^n(A_i f(x_i,y_j))$$

$$\underset{y\in[c,d]}{max} \left| S_1(f)(y) - \int_a^bf(x,y)dx \right| \leq e_1$$
$$\underset{x\in[a,b]}{max} \left| S_2(f)(x) - \int_c^df(x,y)dy \right| \leq e_2$$
$$\left| S(f) - \int_c^d\int_a^b f(x,y) dx \,dy \right| \leq (d-c) e_1 + (b-a) e_2$$
\subsection{Na trójkącie $P$ pole}
formuła rzędu 2:
\begin{center}
$P\cdot f(p_{012})$ gdzie $p_{012} = \frac{1}{3}(p_0+p_1+p_2)$
\end{center}
formuła rzędu 3:
\begin{center}
$\frac{P}{3}\cdot \left[f(p_{01})+f(p_{12})+f(p_{02})\right]$ gdzie $p_{ij} = \frac{1}{2}(p_i+p_j)$
\end{center}
formuła rzędu 4:
\begin{center}
$\frac{P}{60}[ 27f(p_{012})+$\\$ + 3(f(p_0)+f(p_1)+f(p_2)) + 8(f(p_{01})+f(p_{12})+f(p_{02})$\\
\vspace{2mm}
$P = \frac{1}{2}|det[1;\overrightarrow{x};\overrightarrow{y}]|$ - pole
\end{center}
Błąd dla powyższych (przypadek) $E \leq Ch^p$, gdzie $p$ - rząd kwadratur

\subsection{Gaussa-Legendre’a}
Rząd kwadratury równy jest ilości parametrów które ustalamy (dla Gaussa przy liczeniu węzłów od 0 mamy $2n+2$)\\
Wyznaczanie współczynników $A_k$ \\
$$
\int^b_a w(x)l_i(x)dx=A_i, \hspace{3mm} l_i(x)=\prod_{j=0,j\neq i}^{n}\frac{x-x_j}{x_i-x_j}
$$
albo inaczej po ówczesnym obliczeniu $x_0,\dotsc,x_n$:\\
$$
\int^b_a w(x)x^jdx=\sum_{i=0}^nA_ix_i^j \hspace{3mm} \text{dla} \ \ \ j = 0 ,\dotsc, n;
$$
Błąd kw. złożonej Gaussa-Legendre’a:\\
\begin{center}
$(b-a)^5\frac{1}{4320N^4}f^{(4)}(\xi)$
\end{center}

\section{Wielomiany ortogonalne}
$$\langle g,h\rangle= \int_{a}^{b} w(x)g(x)h(x)dx$$
$$P_k (x) = ( \alpha_k x + \beta_k )P_{k-1} (x) + \gamma_k P_{k-2} (x)$$
$$P_{-1} (x) = 0, \hspace{3mm} P_0 (x) = a_0 \neq 0$$
$$\alpha_k = \frac{ a_k }{ a_{k-1} },\hspace{3mm} \beta_k = -\frac{ \alpha_k  \langle xP_{k-1} , P_{k-1} \rangle } { ||P_{k-1}||^2 }$$
$$ \gamma_k = -\frac{ \alpha_k ||P_{k-1}||^2}{ \alpha_{k-1} ||P_{k-2}||^2}$$

\section{Aproksymacja}
Def. Element $f^* \in U$ jest elementem optymalnym dla $f\in V$ ($U \subset V$) względem przestrzeni $U$ $\iff$ dla dowolnego $g \in U$ $\langle f^* -f,g\rangle = 0$ (a więc $f^* -f$ jest ortogonalny do $U$)\\
\vspace{3mm}
ciągła $\langle f,g \rangle = \int_a^b w(x)f(x)g(x) \,dx$\\
dyskretna $\langle f,g \rangle = \sum_{i=0}^Nw_if(x_i)g(x_i)$\\\vspace{3mm}
Konstruujemy macierz $G = \left[\langle g_i,g_j \rangle\right]_{i,k = 0}^n$ (w wierszu nie zmienia się to co na pierwsze w iloczynie) wektor $d = \left[\langle g_i, f_i\rangle \right]_{i = 0}^n$ i szukamy rozwiązania $Ax=d$\\
Dla dyskretnej można $M = \left[f_j(x_i)\right]_{i,k = 0}^n$ (w wierszu te same $x_i$) $G = M^TM$ \ \ \ \  $d = M^TF$ gdzie F to wektor dany z wartościami przybliżanej funkcji
\section{Splajny}
$S'(a)=f'(a)$, $S'(b)=f'(b)$ - warunki Hermite'a\\
$S'(a)=S'(b)$, $S''(a)=S'(b)$ - warunki cykliczne\\
$S''(a)=S''(b)=0$ naturalna funkcja sklejana\\
nota-knot:$
\begin{cases}
    S(\frac{x_0+x_1}{2})=\frac{f(x_0)+f(x_2)}{2}, \\
    S(\frac{x_{n-1}+x_n}{2})=\frac{f(x_{n-1})+f(x_n)}{2},
\end{cases}$
Funkcja sklejana n-tego stopnia to jest, że jest klasy $C^{n-1}$\\
$\Delta_n $ - siatka, podział przedziału[a, b] tż. $a = x_0 < x_1 < ... < x_n = b$.\\
$S_m(\Delta_n, k)$ - przestrzeń funkcji sklejanych, tż.: 1) są klasy $C^k([a, b])$, 2) ich obcięcie do każdego z podprzedziałów $[x_i, x_{i+1}] i = 0, 1, ..., n-1$ jest wielomianem stopnia co najwyżej m.\\
\subsection{$S_1(\Delta_n, 0)$}
$\varphi_0(x) = \frac{x_1-x}{x_1-X_0}, x \in [x_0, x_1]$\\
$\varphi_n(x) = \frac{x-x_{n-1}}{x_n - x_{n-1}}, x \in [x_{n-1}, x_n]$\\
$\varphi_i(x) =\begin{cases} \frac{x-x_{i-1}}{x_i - x_{i-1}}, x \in [x_{i-1}, x_i]\\ \frac{x_{i+1}-x}{x_{i+1} - x_i}, x \in [x_i, x_{i+1}]\end{cases}$ \\
W pozostałych punktach wszystkie funkcje mają wartość 0.\\
\subsection{$S_3(\Delta_n, 2)$}
Węzły równo odległe, $x_i = a+ih$, $h=\frac{b-a}{n}, i =0,...,n$.\\
$$
\frac{1}{h^3}\begin{cases}
(x - x_{i-2})^3 \\%& \text{dla} \hskip2ex x \in [x_{i-2},x_{i-1}]\cap [a, b], \\
h^3 + 3h^2(x - x_{i-1}) + 3h(x - x_{i-1})^2 - 3(x - x_{i-1})^3 \\%& \text{dla} \hskip2ex x \in [x_{i-1},x_i] \cap [a, b], \\
h^3 + 3h^2(x_{i+1} - x) + 3h(x_{i+1} - x)^2 - 3(x_{i+1} - x)^3\\%& \text{dla} \hskip2ex x \in [x_i,x_{i+1}] \cap [a, b], \\
(x_{i+2} - x)^3 \\%& \text{dla} \hskip2ex x \in [x_{i+1},x_{i+2}] \cap [a, b], \\
0 \\%& \text{dla pozostałych} \hskip2ex x,
\end{cases}
$$
Funkcja sklejana: $S(x) = \sum_{k=-1}^{n+1} \alpha_k B_k(x)$,
\section{Metoda iteracji prostej}
Tw. $D$ zbiór domknięty w $\mathbb{R}$ oraz niech $g:D\rightarrow D$ i jest odwzorowaniem zwężającym wtedy metoda iteracji prostej $g$ jest zbieżna na $D$ do dokładnie jednego rozwiązania.\\
Jeśli g jest ciągła i $D$ jest domknięty to mamy\\
$g(x_1) - g(x_0) = g^{\prime}(\xi)(x_1 - x_0)$\\
$g^{\prime}$ to macierz Jakobiego\\
ponadto:\\
$||g(x_1) - g(x_0)|| = ||g^{\prime}(\xi)(x_1 - x_0)|| \leq ||g^{\prime}(\xi)||||(x_1 - x_0)||$
jeśli $||g^{\prime}(\xi)|| < 1$ to mamy odwzorowanie zwężające.\\
Ograniczając z góry każdy wyraz z macierzy $g^{\prime}$ otrzymamy macierz $A$ i zachodzi $||g^{\prime}(\xi)|| < ||A||$ (jeśli ograniczaliśmy $\leq$ analogicznie)
\end{flushleft}
\thispagestyle{empty}
\end{document}
